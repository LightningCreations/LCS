\section{LCS 1 - Definition of the Lightning Creations Specification
Project}\label{lcs-1---definition-of-the-lightning-creations-specification-project}

Publication Date: 09-22-2019

Copyright © 2019 Connor Horman and Lightning Creations. This document
may be freely copied and distributed but changing is not allowed, except
as explicitly licensed.

\subsection{§1 Definition of the
Project}\label{definition-of-the-project}

The Lightning Creations Specification Project is a project which seeks
to publish various Specifications pertaining to Lightning Creations
Projects. The Specifications are published by Lightning Creations and
the LCS Review Committee on behalf of the author of that specification.
These Specifications will be assigned some number, and can be referred
to as LCS followed by the number. For example, the 1st LCS can be
referred to as LCS 1. Additionally, LCS Drafts may be done freely, and
can be referred to as LCSD followed by the name of person who is writing
the Draft, followed by the number that person chose to assign to it.
This will be reviewed for publication according to a process layed out
in a future LCS. This process is herein referred to as the ``LCS Review
Process'', and will be carried out by nominated members of Lightning
Creations, herein referred to as the ``LCS Review Committee''. The
publication of an LCS is not a statement of endorsement of that LCS by
either Lightning Creations or the LCS Review Committee. Neither
Lightning Creations or the LCS Review Committee provides any warranties
of any kind about Published LCS Documents, and may not be held liable
for any damages caused, directly or indirectly, through the use of an
LCS Document.

\subsection{§2 Structure of an LCS
Document}\label{structure-of-an-lcs-document}

An LCS Document or LCS is formatted as a series of sections, where the
\_n\_th section is indicated by a Heading labeled as §\emph{n} followed
by the given name of the section. Additionally, these sections may be
divided into a series of subsections, subsections may be further
subdivided, etc. There is no limit to the depth or number of these
subdivisions. A subdivision, m levels deep, is labeled as
§n1.n2.(\ldots{}).nm. Subdivisions of a section should be related to the
level directly enclosing it. Further, sections of an LCS Document should
be related to the Specification defined in by that LCS Document.

No sections in an LCS Document may be labeled as D§\emph{n}.

There is also a Title, which contains LCS, followed by the assigned LCS
Number, then a Dash, then the chosen title for the specification.

Following the title but before any sections, there is an indication of
the publication date, given in the format mm-dd-yyyy, as well as a short
copyright notice from the author of the Document. If a License
Agreement, which is longer than a few sentences, is given, it is
assigned its own section. There may also be an indication of any
previous LCS Documents which are superseded by the LCS Document. This
may indicate all superseded documents, or only the most recent, if that
most recent one supersedes all the omitted LCS Documents.

\subsubsection{§2.1 Structure of an LCS
Draft}\label{structure-of-an-lcs-draft}

An LCS Draft, or LCSD is formatted similar to an LCS Document. In
addition to the rules above, a Draft Section, labeled as D§\emph{n} may
appear. These sections are part of the draft itself, rather than the
specification, and are not expected to appear in the published LCS
Document. Any Section labeled as §\emph{n} is expected to appear in the
published LCS Document, and are called Standard Sections. If a Draft
Section is changed to be expected to appear in the published LCS
Document, a revision of the draft must be published with the Draft
Section relabeled as a Standard Section. A subdivision of a Standard
Section may be a Draft Section. In this case, the first layered
subdivision which is not expected to appear in the published LCS
Document has D prefix the number of that subdivision. The Standard
Section at the previous layer is said to contain that Draft Section. It
is impossible for a Draft Section to contain a Standard Section in this
way. Instead of a publication date, a draft date is given. An LCS Draft
which has a later draft date then any other LCS Draft with the same
number from the same author

\subsubsection{§2.2 Structure of this
Publication}\label{structure-of-this-publication}

This Publication, referred to as LCS 1, is structured in the same manner
as any LCS Document. This Publication shall be considered an LCS
Document for purposes of review. Additionally, this document shall be
considered an agreement between the nominated members of the LCS Review
Committee. Acceptance of the position as a member of the LCS Review
Committee constitutes acceptance of this agreement.

\subsection{§3 LCS Review Committee}\label{lcs-review-committee}

The LCS Review Committee is a voluntary position appointed by Lightning
Creations. At the offering of a position on the LCS Review Committee,
the member may decline or accept the position. LCS Drafts may be subject
to Independent Review by a member of the LCS Review Committee, possibly
advised by one or more other members, or Committee Review by the entire
LCS Review Committee. The process for Independent Review or Committee
Review is not laid out in this Document, and will be defined by a later
LCS. Prior to publication, a final revision of the LCS Draft, structured
as an LCS Document will be subject to Committee Review. Additionally,
this publication will be subject to Committee Review a year from its
publication date, and updated terms may be published as a new LCS
Document which supersedes this one. The new LCS Document will then come
into effect over this one. A separate LCS Document will be published
will all active members of the LCS Review Committee.
